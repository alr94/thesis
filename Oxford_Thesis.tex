%%%%%%%%%%%%%%%%%%%%%%%%%%%%%%%%%%%%%%%%%%%%%%%%%%%%%%%%%%%%%%%
%% OXFORD THESIS TEMPLATE

% Use this template to produce a standard thesis that meets the Oxford University requirements for DPhil submission
%
% Originally by Keith A. Gillow (gillow@maths.ox.ac.uk), 1997
% Modified by Sam Evans (sam@samuelevansresearch.org), 2007
% Modified by John McManigle (mcmanigle@gmail.com), 2015

% I've (John) tried to comment this file extensively, so read through it to see how to use the various options.  Remember
% that in LaTeX, any line starting with a % is NOT executed.  Several places below, you have a choice of which line to use
% out of multiple options (eg draft vs final, for PDF vs for binding, etc.)  When you pick one, add a % to the beginning of
% the lines you don't want.


%%%%% CHOOSE PAGE LAYOUT
% The most common choices should be below.  You can also do other things, like replacing "a4paper" with "letterpaper", etc.

% This one will format for two-sided binding (ie left and right pages have mirror margins; blank pages inserted where needed):
\documentclass[a4paper,twoside]{ociamthesis}
% This one will format for one-sided binding (ie left margin > right margin; no extra blank pages):
%\documentclass[a4paper]{ociamthesis}
% This one will format for PDF output (ie equal margins, no extra blank pages):
%\documentclass[a4paper,nobind]{ociamthesis} 



%%%%% SELECT YOUR DRAFT OPTIONS
% Three options going on here; use in any combination.  But remember to turn the first two off before
% generating a PDF to send to the printer!

% This adds a "DRAFT" footer to every normal page.  (The first page of each chapter is not a "normal" page.)
\fancyfoot[C]{\emph{DRAFT Printed on \today}}  

% This highlights (in blue) corrections marked with (for words) \mccorrect{blah} or (for whole
% paragraphs) \begin{mccorrection} . . . \end{mccorrection}.  This can be useful for sending a PDF of
% your corrected thesis to your examiners for review.  Turn it off, and the blue disappears.
\correctionstrue


%%%%% BIBLIOGRAPHY SETUP
% Note that your bibliography will require some tweaking depending on your department, preferred format, etc.
% The options included below are just very basic "sciencey" and "humanitiesey" options to get started.
% If you've not used LaTeX before, I recommend reading a little about biblatex/biber and getting started with it.
% If you're already a LaTeX pro and are used to natbib or something, modify as necessary.
% Either way, you'll have to choose and configure an appropriate bibliography format...

% The science-type option: numerical in-text citation with references in order of appearance.
\usepackage[style=numeric-comp, sorting=none, backend=biber, doi=false, isbn=false]{biblatex}
\newcommand*{\bibtitle}{References}

% The humanities-type option: author-year in-text citation with an alphabetical works cited.
%\usepackage[style=authoryear, sorting=nyt, backend=biber, maxcitenames=2, useprefix, doi=false, isbn=false]{biblatex}
%\newcommand*{\bibtitle}{Works Cited}

% This makes the bibliography left-aligned (not 'justified') and slightly smaller font.
\renewcommand*{\bibfont}{\raggedright\small}

% Change this to the name of your .bib file (usually exported from a citation manager like Zotero or EndNote).
\addbibresource{references.bib}


% Uncomment this if you want equation numbers per section (2.3.12), instead of per chapter (2.18):
%\numberwithin{equation}{subsection}



%%%%% THESIS / TITLE PAGE INFORMATION
% Everybody needs to complete the following:
\title{Evaluating the Performance of the ProtoDUNE--SP Detector using Michel Electrons}
\author{Aidan Reynolds}
\college{University College}

% Master's candidates who require the alternate title page (with candidate number and word count)
% must also un-comment and complete the following three lines:
%\masterssubmissiontrue
%\candidateno{933516}
%\wordcount{28,815}

% Uncomment the following line if your degree also includes exams (eg most masters):
%\renewcommand{\submittedtext}{Submitted in partial completion of the}
% Your full degree name.  (But remember that DPhils aren't "in" anything.  They're just DPhils.)
\degree{Doctor of Philosophy}
% Term and year of submission, or date if your board requires (eg most masters)
\degreedate{Hilary 2020}


%%%%% YOUR OWN PERSONAL MACROS
% This is a good place to dump your own LaTeX macros as they come up.
\newcommand{\fig}[1]{Fig. \ref{#1}}

% To make text superscripts shortcuts
	\renewcommand{\th}{\textsuperscript{th}} % ex: I won 4\th place
	\newcommand{\nd}{\textsuperscript{nd}}
	\renewcommand{\st}{\textsuperscript{st}}
	\newcommand{\rd}{\textsuperscript{rd}}



%%%%% THE ACTUAL DOCUMENT STARTS HERE
\begin{document}



%%%%% CHOOSE YOUR LINE SPACING HERE
% This is the official option.  Use it for your submission copy and library copy:
\setlength{\textbaselineskip}{22pt plus2pt}
% This is closer spacing (about 1.5-spaced) that you might prefer for your personal copies:
%\setlength{\textbaselineskip}{18pt plus2pt minus1pt}

% You can set the spacing here for the roman-numbered pages (acknowledgements, table of contents, etc.)
\setlength{\frontmatterbaselineskip}{17pt plus1pt minus1pt}

% Leave this line alone; it gets things started for the real document.
\setlength{\baselineskip}{\textbaselineskip}


%%%%% CHOOSE YOUR SECTION NUMBERING DEPTH HERE
% You have two choices.  First, how far down are sections numbered?  (Below that, they're named but
% don't get numbers.)  Second, what level of section appears in the table of contents?  These don't have
% to match: you can have numbered sections that don't show up in the ToC, or unnumbered sections that
% do.  Throughout, 0 = chapter; 1 = section; 2 = subsection; 3 = subsubsection, 4 = paragraph...

% The level that gets a number:
\setcounter{secnumdepth}{3}
% The level that shows up in the ToC:
\setcounter{tocdepth}{3}


%%%%% ABSTRACT SEPARATE
% This is used to create the separate, one-page abstract that you are required to hand into the Exam
% Schools.  You can comment it out to generate a PDF for printing or whatnot.
\begin{abstractseparate}
	This thesis presents the results of a study of electromagnetic interactions in 
the \protodune{} liquid argon time projection chamber (LArTPC) detector. The 
LArTPC detector technology provides high spatial resolution on the final states 
of neutrino interactions, allowing interaction modes to be distinguished based 
on the geometry of the interactions in the event. In order to perform high
precision measurements of neutrinos in LArTPC detectors, final state particles
need to be effectively identified, and their energy accurately reconstructed.
This work focussed on these challenges with two studies on data from the
\protodune{} LArTPC, a study of track--shower classification, and a study of 
Michel electron energy reconstruction. A track--shower classification algorithm
is developed based on the use of convolutional neural networks, and its
performance is compared to the current track--shower classification algorithm.
Michel electrons are used as a source of electromagnetic activity in the tens of
MeV range, and the energy resolution and bias for these electrons are estimated.

% In
% this work, track--shower discrimination and Michel electron reconstruction are 
% studied in the \protodune{} LArTPC. 
% 
% A track--shower discriminator based on
% convolutional neural networks was developed, and shows a 
% 
% A convolutional neural network is trained 
% for track--shower classification, 
% 
%     which demonstrates a significant improvement
% over the current algorithms in terms of 
% 
% 
% In order to perform high 
% precision measurements of supernova neutrinos in LArTPC detectors, electrons 
% must be identified and their energy accurately reconstructed. In this work EM 
% activity is studied in the 10--50 MeV range using Michel electrons as a source 
% with a well defined energy spectrum. The energy uncertainty and bias for
% reconstructed Michel electron events in the \protodune{} LArTPC are estimated.
% 
% The sensitivity, bias, and energy scale are 
% studied and the implications for neutrino physics in the Deep Underground 
% Neutrino Experiment are discussed.
 % Create an abstract.tex file in the 'text' folder for your abstract.
\end{abstractseparate}


% JEM: Pages are roman numbered from here, though page numbers are invisible until ToC.  This is in
% keeping with most typesetting conventions.
\begin{romanpages}

% Title page is created here
\maketitle

%%%%% DEDICATION -- If you'd like one, un-comment the following.
%\begin{dedication}
%This thesis is dedicated to\\
%someone\\
%for some special reason\\
%\end{dedication}

%%%%% ACKNOWLEDGEMENTS -- Nothing to do here except comment out if you don't want it.
\begin{acknowledgements}
 	This thesis would not have been possible without the advice and support of many
people. I would like to start by thanking my supervisor, Alfons Weber. He has
been a brilliant advisor, and I would like to thank him in particular for all
the opportunities he has given me. Thanks also to the rest of the Oxford
Neutrino Physics group, particularly Justo Martin--Albo whose support during 
my first year helped me to settle into the DPhil, and my fellow students -- 
Fabio, Alex, and Ciaran -- who have made the office, both physical and 
virtual, an enjoyable place to work.

\bigskip\noindent
I have worked with many great collaborators from the DUNE experiment
as part of my DPhil. I'd particularly like to thank the members of the 
\protodune{} reconstruction and analysis group who have always given 
excellent feedback on my work. Special thanks go to Dorota Stefan and Robert 
Sulej, who were incredibly supportive during the early years of my DPhil, and 
to Leigh Whitehead and Tingjun Yang, whose insightful conversations have been 
invaluable. 

\bigskip\noindent
I am very fortunate to have had the chance to work at CERN for part of my DPhil,
and I would like to thank all of the members of the on--site \protodune{} 
team. Especially Alex, Chris, Geoff, James, Milo, and Seb, who I thoroughly 
enjoyed working with during my time at CERN, and who I have learned so much 
from. 

\bigskip\noindent
Finally, my greatest thanks go to my friends and family, for their continued
support over the years. To Amy, Rory, and Helena, who have always been there to
help me relax at the end of the day; to the members of Oxford Ultimate, who have
been so welcoming over the past year, you have kept me going during the final 
stages; to my family, who fostered a love of learning, which will always be 
with me; and to Ellie, who could never understand how much her support has 
meant to me over the years.

\end{acknowledgements}

%%%%% ABSTRACT -- Nothing to do here except comment out if you don't want it.
\begin{abstract}
	This thesis presents the results of a study of electromagnetic interactions in 
the \protodune{} liquid argon time projection chamber (LArTPC) detector. The 
LArTPC detector technology provides high spatial resolution on the final states 
of neutrino interactions, allowing interaction modes to be distinguished based 
on the geometry of the interactions in the event. In order to perform high
precision measurements of neutrinos in LArTPC detectors, final state particles
need to be effectively identified, and their energy accurately reconstructed.
This work focussed on these challenges with two studies on data from the
\protodune{} LArTPC, a study of track--shower classification, and a study of 
Michel electron energy reconstruction. A track--shower classification algorithm
is developed based on the use of convolutional neural networks, and its
performance is compared to the current track--shower classification algorithm.
Michel electrons are used as a source of electromagnetic activity in the tens of
MeV range, and the energy resolution and bias for these electrons are estimated.

% In
% this work, track--shower discrimination and Michel electron reconstruction are 
% studied in the \protodune{} LArTPC. 
% 
% A track--shower discriminator based on
% convolutional neural networks was developed, and shows a 
% 
% A convolutional neural network is trained 
% for track--shower classification, 
% 
%     which demonstrates a significant improvement
% over the current algorithms in terms of 
% 
% 
% In order to perform high 
% precision measurements of supernova neutrinos in LArTPC detectors, electrons 
% must be identified and their energy accurately reconstructed. In this work EM 
% activity is studied in the 10--50 MeV range using Michel electrons as a source 
% with a well defined energy spectrum. The energy uncertainty and bias for
% reconstructed Michel electron events in the \protodune{} LArTPC are estimated.
% 
% The sensitivity, bias, and energy scale are 
% studied and the implications for neutrino physics in the Deep Underground 
% Neutrino Experiment are discussed.

\end{abstract}

%%%%% MINI TABLES
% This lays the groundwork for per-chapter, mini tables of contents.  Comment the following line
% (and remove \minitoc from the chapter files) if you don't want this.  Un-comment either of the
% next two lines if you want a per-chapter list of figures or tables.
\dominitoc % include a mini table of contents
%\dominilof  % include a mini list of figures
%\dominilot  % include a mini list of tables

% This aligns the bottom of the text of each page.  It generally makes things look better.
\flushbottom

% This is where the whole-document ToC appears:
\tableofcontents

\listoffigures
	\mtcaddchapter
% \mtcaddchapter is needed when adding a non-chapter (but chapter-like) entity to avoid confusing minitoc

% Uncomment to generate a list of tables:
%\listoftables
%	\mtcaddchapter

%%%%% LIST OF ABBREVIATIONS
% This example includes a list of abbreviations.  Look at text/abbreviations.tex to see how that file is
% formatted.  The template can handle any kind of list though, so this might be a good place for a
% glossary, etc.
\begin{mclistof}{List of Abbreviations}{3.2cm}
	\item [ Adam       ] {Adaptive Momentum Estimation}
	\item [ ANN        ] {Artificial Neural Network}
	\item [ APA        ] {Anode--plane Assembly}
	\item [ BI         ] {Beam Instrumentation}
	\item [ CC         ] {Charged Current}
	\item [ CE         ] {Cold Electronics}
	\item [ CNN        ] {Convolutional Neural Network}
	\item [ CP         ] {Charge--Parity}
	\item [ CPT        ] {Charge--Parity--Time}
	\item [ CPA        ] {Cathode--plane Assembly}
	\item [ CRT        ] {Cosmic--ray Tagger}
	\item [ CTB        ] {Central Trigger Board}
	\item [ DAQ        ] {Data Acquisition System}
	\item [ DIS        ] {Deep Inelastic Scattering}
	\item [ DUNE       ] {Deep Underground Neutrino Experiment}
	\item [ ES         ] {Elastic Scattering}
	\item [ FELIX      ] {Front--end Link Exchange}
	\item [ FEMB       ] {Front--end Mother Board}
	\item [ FFT        ] {Fast Fourier Transform}
	\item [ FNAL       ] {Fermilab National Accelerator Laboratory}
	\item [ IO         ] {Inverted Ordering}
	\item [ KamLAND    ] {Kamioka Liquid Scintillator Anti--neutrino Detector}
	\item [ LArTPC     ] {Liquid Argon Time Projection Chamber}
	\item [ LEP        ] {Large Electron--Positron Collider}
	\item [ LHC        ] {Large Hadron Collider}
	\item [ LIGO       ] {Laser Interferometer Gravitational--Wave Observatory}
	\item [ LNGS       ] {Laboratori Nazionali del Gran Sasso}
	\item [ MC         ] {Monte--carlo Simulation}
	\item [ MIP        ] {Minimum Ionising Particle}
	\item [ ML         ] {Machine Learning}
	\item [ MLP        ] {Multi--layer Perceptron}
	\item [ MSW        ] {Mikheyev--Smirnov--Wolfenstein}
	\item [ NC         ] {Neutral Current}
	\item [ NN         ] {Neural Network}
	\item [ NO         ] {Normal Ordering}
	\item [ OM         ] {Online Monitoring}
	\item [ PDS        ] {Photon Detection System}
	\item [ PFParticle ] {Paricle Flow Particle}
	\item [ PID        ] {Particle Identification}
	\item [ PMNS       ] {Pontecorvo--Maki--Nakagawa--Sakata}
	\item [ QE         ] {Quasi--elastic}
	\item [ RCE        ] {Reconfigurable Computing Element}
	\item [ ReLU       ] {Rectified Linear Unit}
	\item [ RES        ] {Resonance}
	\item [ ResNet     ] {Residual Neural Network}
	\item [ RMS        ] {Root Mean Square Deviation}
	\item [ ROC        ] {Reciever Operator Characteristic}
	\item [ SCE        ] {Space Charge Effect}
	\item [ SGD        ] {Stochastic Gradient Descent}
	\item [ SiPM       ] {Silicon Photo--multiplier}
	\item [ SM         ] {Standard Model}
	\item [ SNO        ] {Sudbury Neutrino Observatory}
	\item [ SP         ] {Single Phase}
	\item [ SSM        ] {Standard Solar Model}
	\item [ SSP        ] {SiPM Signal Processor}
	\item [ SURF       ] {Sanford Underground Research Facility}
	\item [ TOF        ] {Time of Flight}
	\item [ TPB        ] {Tetraphenyl-butadiene}
	\item [ TSE        ] {Track--Shower--Empty}
	\item [ WIB        ] {Warm Interface Board}
\end{mclistof} 


% The Roman pages, like the Roman Empire, must come to its inevitable close.
\end{romanpages}


%%%%% CHAPTERS
% Add or remove any chapters you'd like here, by file name (excluding '.tex'):
\flushbottom
% \begin{savequote}[8cm]
% \textlatin{Neque porro quisquam est qui dolorem ipsum quia dolor sit amet, consectetur, adipisci velit...}
% 
% There is no one who loves pain itself, who seeks after it and wants to have it, simply because it is pain...
%   \qauthor{--- Cicero's \textit{de Finibus Bonorum et Malorum}}
% \end{savequote}

\chapter{\label{ch:1-intro}Introduction} 

\minitoc


% \begin{savequote}[8cm]
% \textlatin{Neque porro quisquam est qui dolorem ipsum quia dolor sit amet, consectetur, adipisci velit...}
% 
% There is no one who loves pain itself, who seeks after it and wants to have it, simply because it is pain...
%   \qauthor{--- Cicero's \textit{de Finibus Bonorum et Malorum}}
% \end{savequote}

\chapter{\label{ch:2-neutrinophysics}Neutrino Physics} 

\minitoc

\section{Neutrinos in the Standard Model}

\section{Neutrino Oscillations}

\section{Supernova Neutrinos}

\section{Neutrino Detection}

\section{The Deep Underground Neutrino Experiment}


% \begin{savequote}[8cm]
% \textlatin{Neque porro quisquam est qui dolorem ipsum quia dolor sit amet, consectetur, adipisci velit...}
% 
% There is no one who loves pain itself, who seeks after it and wants to have it, simply because it is pain...
%   \qauthor{--- Cicero's \textit{de Finibus Bonorum et Malorum}}
% \end{savequote}

\chapter{\label{ch:3-protodune}The ProtoDUNE--SP Detector} 

\minitoc

\section{Liquid Argon Time Projection Chambers}

\section{The ProtoDUNE--SP LArTPC}

\section{The H4 Beam Line}

\section{Cosmic Rays at ProtoDUNE--SP}

\section{The ProtoDUNE--SP Online Monitoring System}


% \begin{savequote}[8cm]
% \textlatin{Neque porro quisquam est qui dolorem ipsum quia dolor sit amet, consectetur, adipisci velit...}
% 
% There is no one who loves pain itself, who seeks after it and wants to have it, simply because it is pain...
%   \qauthor{--- Cicero's \textit{de Finibus Bonorum et Malorum}}
% \end{savequote}

\chapter{\label{ch:4-energyloss}Energy Loss in Liquid Argon} 

\minitoc

\section{Electron Energy Loss}

\section{Photon Energy Loss}

\section{Implicatons on Electron Reconstruction im Liquid Argon}


% \begin{savequote}[8cm]
% \textlatin{Neque porro quisquam est qui dolorem ipsum quia dolor sit amet, consectetur, adipisci velit...}
% 
% There is no one who loves pain itself, who seeks after it and wants to have it, simply because it is pain...
%   \qauthor{--- Cicero's \textit{de Finibus Bonorum et Malorum}}
% \end{savequote}

\chapter{\label{ch:5-michel}Study of Michel Electrons in ProtoDUNE--SP} 

\minitoc

\section{Electromagnetic Energy Loss in Liquid Argon at 0--50 MeV}

\section{Michel Electron Event Selection}

\section{Michel Electron Energy Reconstruction}

\section{Reconstructed Michel Electron Spectrum}

\section{Conclusions}


% \begin{savequote}[8cm]
% \textlatin{Neque porro quisquam est qui dolorem ipsum quia dolor sit amet, consectetur, adipisci velit...}
% 
% There is no one who loves pain itself, who seeks after it and wants to have it, simply because it is pain...
%   \qauthor{--- Cicero's \textit{de Finibus Bonorum et Malorum}}
% \end{savequote}

\chapter{\label{ch:6-implications}Implicatons for DUNE} 

\minitoc

\section{Neutrino Oscillation Physics}

\section{Supernova Neutrino Physics}

% \begin{savequote}[8cm]
% \textlatin{Neque porro quisquam est qui dolorem ipsum quia dolor sit amet, consectetur, adipisci velit...}
% 
% There is no one who loves pain itself, who seeks after it and wants to have it, simply because it is pain...
%   \qauthor{--- Cicero's \textit{de Finibus Bonorum et Malorum}}
% \end{savequote}

\chapter{\label{ch:7-conclusions}Conclusions} 

\minitoc




%% APPENDICES %% 
% Starts lettered appendices, adds a heading in table of contents, and adds a
%    page that just says "Appendices" to signal the end of your main text.
\startappendices
% Add or remove any appendices you'd like here:
% \begin{savequote}[8cm]
% \textlatin{Cor animalium, fundamentum e\longs t vitæ, princeps omnium, Microco\longs mi Sol, a quo omnis vegetatio dependet, vigor omnis \& robur emanat.}
% 
% The heart of animals is the foundation of their life, the sovereign of everything within them, the sun of their microcosm, that upon which all growth depends, from which all power proceeds.
%   \qauthor{--- William Harvey \cite{harvey_exercitatio_1628}}
% \end{savequote}

\chapter{\label{app:1-example}Example Apendix}

\minitoc

\section{Example Apendix Title}

\label{sec:example}

Some example text.



%%%%% REFERENCES

% JEM: Quote for the top of references (just like a chapter quote if you're using them).  Comment to skip.
% \begin{savequote}[8cm]
% The first kind of intellectual and artistic personality belongs to the hedgehogs, the second to the foxes \dots
%   \qauthor{--- Sir Isaiah Berlin \cite{berlin_hedgehog_2013}}
% \end{savequote}

\setlength{\baselineskip}{0pt} % JEM: Single-space References

{\renewcommand*\MakeUppercase[1]{#1}%
\printbibliography[heading=bibintoc,title={\bibtitle}]}


\end{document}
