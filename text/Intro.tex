\chapter{\label{ch:intro}Introduction} 

\minitoc

Since the discovery of neutrino flavour oscillations, which implies that
neutrinos have mass, neutrino physics has enjoyed a period of rapid development.
The field has begun to transition into an era of precision, with many of the
parameters governing these oscillations having been well constrained. The fact
that neutrinos have mass, and the success of the
Pontecorvo--Maki--Nakagawa--Sakata (PMNS) theory in describing neutrino 
oscillations, leads to a number of fundamental questions which have important 
implications for both particle physics and cosmology: 
\begin{itemize}
	\item What is the mechanism giving rise to neutrino mass? 
	\item Are neutrinos Dirac or Majorana particles?
	\item What is the absolute scale and ordering of the neutrino masses?
	\item Do neutrinos and anti--neutrinos oscillate differently, and would this 
	      help to explain the matter anti--matter asymmetry in the universe?
	\item Are there any sterile neutrinos?
\end{itemize}

In addition to these questions, the high resolution and large masses of modern 
neutrino detectors make them useful tools for both astronomy and astrophysics. 
2017 has widely been considered as the dawn of multi--messenger astronomy, 
with a measurement of gravitational waves at the Laser Interferometer 
Gravitational--Wave Observatory (LIGO) being correlated with measurements of a 
neutron star merger from electromagnetic telescopes\cite{Abbott2017}.  This 
measurement was shortly followed by a similar correlation but in the neutrino 
sector between a high energy neutrino event in the IceCube Neutrino 
Observatory and a number of traditional telescopes\cite{Aartsen2018}.  Within 
our galaxy, neutrino detectors provide a unique opportunity to understand the 
underlying mechanisms in supernovae; in the case of such a supernova, the 
structure of the neutrino flux at earth provides a mechanism to measure 
effects in the early stages of the supernova burst, which are inaccessible 
with electromagnetic measurements\cite{Scholberg:2012id}.

Each of these questions places unique constraints on the design of an
appropriate neutrino detector. The discovery of a matter anti--matter asymmetry
in neutrino oscillations could be answered by making precise measurements of
neutrino oscillations. This requires reliably identifying the flavour and energy
of neutrinos in order to measure the appearance and disappearance spectra
associated with neutrinos produced in long baseline neutrino experiments. To
identify the low energy electrons produced in supernova neutrino interactions, a
detector with low thresholds and low backgrounds is required. The Deep
Underground Neutrino Experiment (DUNE) aims to tackle these challenges by
utilising the Liquid Argon Time Projection Chamber (LArTPC) technology, whose
high spatial and calorimetric resolution allows for more accurate geometric
classification of neutrino interactions. To achieve these goals, a significant 
programme of LArTPC research is ongoing with construction, reconstruction, and 
analysis methods all under development in a number of LArTPC based 
experiments\cite{Acciarri:2016smi, Cavanna:2014iqa, Antonello:2015lea, 
Abi:2017aow}. 

This thesis presents analyses of charged particle interactions in the
\protodune{} LArTPC detector, using data collected during a test beam run of the
detector between August and November 2018. Hit classification and Michel 
electron reconstruction are investigated, and a sample of Michel electrons is 
used to provide a measurement of the energy resolution and bias for low energy 
electrons in \protodune{}.

Particle classification plays an important role in event reconstruction in 
LArTPC detectors. In particular, the clustering of hits into tracks and 
showers is an important step in reconstructing events in a LArTPC. After tracks
and showers have been reconstructed, they can be combined to build up a picture
of the full particle interaction. In this thesis, we present the results of the 
development of a hit classification algorithm based on convolutional neural
networks. The primary goal of this algorithm is to classify hits as 
track--like or shower--like on a hit--by--hit basis. The output from the hit 
classification has been applied to analyses of beam particle interactions, 
such as pion cross--section analyses, and cosmic--ray interactions, which 
are used to calibrate the \protodune{} detector. It is also used to select a 
sample of Michel electron candidates, which are analysed in this thesis.

Michel electrons have an energy spectrum spanning 0--60 MeV. Understanding
electrons in this energy range is important, as they have a similar energy to 
the electrons produced when neutrinos from supernova bursts interact. In a 
LArTPC at these energies the energy deposition of electrons transitions 
between ionisation dominated and radiation dominated regimes, making for a 
unique event signature. The work presented in this thesis details the 
selection and reconstruction of Michel electron events in \protodune{}, based 
on machine learning algorithms. Analysis of the reconstructed Michel electron 
events will be used to quantify the energy resolution and bias for low energy 
electrons in \protodune{} based on this approach. The results of this analysis 
provide valuable inputs to studies of supernova burst neutrinos in LArTPC 
detectors.

In this thesis, chapter \ref{ch:neutrinophysics} provides a theoretical 
overview of neutrinos within the standard model. Interactions, oscillations, 
and production will be discussed summarising the current knowledge in this
field, as well as open questions which will be studied in ongoing and upcoming 
experiments. The role of neutrinos in supernova bursts and the detection of 
such neutrinos in a LArTPC detector will also be discussed in more detail.

The ProtoDUNE--SP experiment is described in Chapter \ref{ch:protodune},
including details of the beam line, detector, cosmic--ray flux, and simulations.
An overview of the LArTPC detection principle will be given with specific
details of the ProtoDUNE--SP design. Some details of detector operations will be
discussed, paying particular attention to the monitoring of the detector via the
online data quality monitoring system, which has been the author's major 
contribution to the detector operations.

Chapter \ref{ch:energyloss} will cover details of electromagnetic energy loss
in liquid argon. Electron and photon energy loss will be discussed as well as
processes leading to electron-ion recombination. The impacts of these effects on
electron reconstruction in liquid argon will be highlighted.

The main analyses of this thesis rely on the use of neural network algorithms
for reconstruction, therefore, Chapter \ref{ch:ml} with briefly outline the
relevant details of these algorithms.

Chapters \ref{ch:chargeid} and \ref{ch:michel}, will detail the main analyses 
of this thesis. Details of a hit classification algorithm based on 
convolutional neural networks will be given, and Michel electron 
reconstruction will be highlighted as an example use for the output of this 
algorithm. Michel electron production and energy loss in liquid argon will be 
discussed. This will be followed by details of the reconstruction strategy 
used in the Michel electron analysis. The reconstructed Michel electron 
spectrum will be compared between data and simulation, and the energy 
resolution and bias for low energy electrons in the ProtoDUNE--SP detector 
will be estimated. 

A summary of the results will be given in Chapter \ref{ch:conclusion} along 
with a discussion of the implications of these results for physics in LArTPC 
detectors.
