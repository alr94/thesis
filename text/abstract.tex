This thesis presents the results of a study of electromagnetic interactions in 
the \protodune{} liquid argon time projection chamber (LArTPC) detector. The 
LArTPC detector technology provides high spatial resolution on the final states 
of charged particle interactions, allowing different interaction modes to be 
distinguished based on the geometry of the ionisation energy deposition in the 
event, as a result, the technology has been used in a number of recent neutrino
experiments, and is the technology of choice for the Deep Underground Neutrino 
Experiment. In order to perform high precision measurements of neutrinos in 
LArTPC detectors, final state particles need to be effectively identified, and 
their energy accurately reconstructed. This work focusses on these challenges 
with two studies on data from the \protodune{} LArTPC: a study of 
track--shower classification, and a study of Michel electron energy 
reconstruction. A track--shower classification algorithm is developed based on 
the use of convolutional neural networks, and its performance is compared to 
the current track--shower classification algorithm in \protodune{}. The results
of this network were used to select a sample of Michel electron events, which
were used to develop a Michel electron reconstruction algorithm based on 
semantic segmentation with a convolutional neural network. This sample of 
Michel electron events is used to estimate the energy resolution and bias for 
low--energy electrons in \protodune{} based on this algorithm.
