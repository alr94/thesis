This thesis presents the results of a study of electromagnetic interactions in 
the \protodune{} liquid argon time projection chamber (LArTPC) detector. The 
LArTPC detector technology provides high spatial resolution on the final states 
of neutrino interactions, allowing interaction modes to be distinguished based 
on the geometry of the interactions in the event. In order to perform high
precision measurements of neutrinos in LArTPC detectors, final state particles
need to be effectively identified, and their energy accurately reconstructed.
This work focussed on these challenges with two studies on data from the
\protodune{} LArTPC, a study of track--shower classification, and a study of 
Michel electron energy reconstruction. A track--shower classification algorithm
is developed based on the use of convolutional neural networks, and its
performance is compared to the current track--shower classification algorithm.
Michel electrons are used as a source of electromagnetic activity in the tens of
MeV range, and the energy resolution and bias for these electrons are estimated.

% In
% this work, track--shower discrimination and Michel electron reconstruction are 
% studied in the \protodune{} LArTPC. 
% 
% A track--shower discriminator based on
% convolutional neural networks was developed, and shows a 
% 
% A convolutional neural network is trained 
% for track--shower classification, 
% 
%     which demonstrates a significant improvement
% over the current algorithms in terms of 
% 
% 
% In order to perform high 
% precision measurements of supernova neutrinos in LArTPC detectors, electrons 
% must be identified and their energy accurately reconstructed. In this work EM 
% activity is studied in the 10--50 MeV range using Michel electrons as a source 
% with a well defined energy spectrum. The energy uncertainty and bias for
% reconstructed Michel electron events in the \protodune{} LArTPC are estimated.
% 
% The sensitivity, bias, and energy scale are 
% studied and the implications for neutrino physics in the Deep Underground 
% Neutrino Experiment are discussed.
