This thesis presents the results of a study of electromagnetic interactions in 
the \protodune{} liquid argon time projection chamber (LArTPC) detector. The 
LArTPC detector technology provides high spatial resolution on the final states 
of charged particle interactions, allowing different interaction modes to be 
distinguished based on the geometry of the ionisation energy deposition in the 
event. In order to perform high precision measurements of neutrinos in LArTPC 
detectors, final state particles need to be effectively identified, and their 
energy accurately reconstructed. This work focussed on these challenges with 
two studies on data from the \protodune{} LArTPC: a study of track--shower 
classification, and a study of Michel electron energy reconstruction. A 
track--shower classification algorithm is developed based on the use of 
convolutional neural networks, and its performance is compared to the current 
track--shower classification algorithm. A Michel electron reconstruction
algorithm is developed based on semantic segmentation with a convolutional
neural network, and a sample of Michel electron events is used to estimate the
energy resolution and bias for low--energy electrons in \protodune{} based on
this algorithm.
