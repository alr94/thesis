This thesis presents the results of a study of electromagnetic interactions in 
the \protodune{} liquid argon time projection chamber (LArTPC) detector. The 
LArTPC detector technology provides high spatial resolution on the final states 
of charged particle interactions, which allows for different interaction modes 
to be distinguished based on the geometry of the ionisation energy deposition 
in the event. As a result, the technology has been used in a number of recent 
neutrino experiments, and is the technology of choice for the Deep Underground 
Neutrino Experiment. In order to perform high precision measurements of 
neutrinos in LArTPC detectors, final state particles need to be effectively 
identified, and their energy accurately reconstructed. This thesis focusses on 
these challenges with two studies on data from the \protodune{} LArTPC: a 
study of track--shower classification, and a study of Michel electron energy 
reconstruction. A track--shower classification algorithm was developed based 
on the use of convolutional neural networks, and its performance was compared 
to the current track--shower classification algorithm in \protodune{}. The 
results of this network were used to select a sample of Michel electron 
events, which were then used to develop a Michel electron reconstruction 
algorithm based on semantic segmentation with a convolutional neural network. 
This sample of Michel electron events were then used to estimate the energy 
resolution and bias for low--energy electrons in \protodune{} based on this 
algorithm.
