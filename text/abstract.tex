This thesis presents the results of the study of electromagnetic (EM)
	interactions in the ProtoDUNE--SP liquid argon time projection chamber
	(LArTPC) detector. The LArTPC detector technology provides high spatial
	resolution on the final states of neutrino interactions, allowing interaction 
	modes to be distinguished based on the event topology. In order to perform
	high precision measurements of \(\nu_e\) in LArTPC detectors, electrons must
	be identified and their energy accurately reconstructed. In this work EM
	activity is studied in the 10--50 MeV range using Michel electrons as a source
	with a well defined energy spectrum. The sensitivity, bias and energy scale
	are studied and the implications for neutrino physics in the Deep Underground
	Neutrino Experiment (DUNE) are discussed.
