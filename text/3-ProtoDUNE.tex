\chapter{\label{ch:3-protodune}The ProtoDUNE--SP Experiment} 

\minitoc

%%%%%%%%%%%%%%%%%%%%%%%%%%%%%%%%%%%%%%%%%%%%%%%%%%%%%%%%%%%%%%%%%%%%%%%%%%%%%%%%
% From COS
%
% This chapter will discuss the ProtoDUNE--SP experiment and it's role in the
% development of the proposed DUNE experiment. The LArTPC technology will be
% detailed in the general case and then the specifics of the ProtoDUNE--SP
% detector will be given. Details of the major particle fluxes in ProtoDUNE--SP
% will be outlined, along with a discussion of the simulation and reconstruction
% of each flux. Finally, as my main contribution to detector operations during
% data taking was developing for the ProtoDUNE--SP online monitoring system, this
% will be discussed in more depth. 
% 
% The work for the online monitoring subsection has been completed as part of my
% duties as an on--site expert at CERN. I expect to be able to complete the rest
% of the work by the end of December 2019 alongside the other analysis work.
%%%%%%%%%%%%%%%%%%%%%%%%%%%%%%%%%%%%%%%%%%%%%%%%%%%%%%%%%%%%%%%%%%%%%%%%%%%%%%%%
% TODO 
% image of experiment
% explain scintialltion process
%%%%%%%%%%%%%%%%%%%%%%%%%%%%%%%%%%%%%%%%%%%%%%%%%%%%%%%%%%%%%%%%%%%%%%%%%%%%%%%%


\protodune{} is one of two prototypes for the DUNE far detector modules that has
been operating at the Neutrino Platform at CERN since the summer of 2018. The
experiment collected data from a charged particle beam for approximately 3 
months before Long Shutdown 2 of the Large Hadron Collider. Since then a 
programme of cosmic ray data collection has been ongoing.

This chapter will outline the technical details of the \protodune{} experiment.
Section \ref{sec:pdsp_dune} will outline the role of \protodune{} in the context
of the DUNE experiment. This will be followed by a discussion of the main
elements of the experiment, the \protodune{} detector systems and the H4
beam line, in Sections \ref{sec:pdsp_detector} and \ref{sec:h4} respectively.
The high event rate in \protodune{} is dominated by a high flux of cosmic rays
which will be discussed in Section \ref{sec:pdsp_cosmic}. Section
\ref{sec:pdsp_sim_reco} will then discuss the simulation and reconstruction of 
\protodune{} data. Finally Section \ref{sec:pdsp_om} will cover details of the 
online monitoring system in \protodune{}; as the primary developer and expert 
on the \protodune{} online monitoring system during my time at CERN, the 
development and maintenance of this system represent a significant body of 
work over 12 months.  

\section{The DUNE Experiment and \protodune{}} \label{sec:pdsp_dune}

\begin{figure}

	\centering

	\includegraphics[width=\textwidth]{figures/dune_baseline.png}

	\caption
	[The Deep Underground Neutrino Experiment.]
	{The Deep Underground Neutrino Experiment. Figure from \cite{TODO}.}

	\label{fig:dune_baseline}

\end{figure}

The DUNE experiment will be a next generation neutrino physics and nucleon decay
experiment consisting of three principal components; an intense broad band 
neutrino beam and precise near detector based at the Fermilab National 
Accelerator Laboratory near Chicago, and a far detector at Sanford Underground 
Research Facility in South Dakota, approximately 1300 km away from the 
neutrino source, as demonstrated in Figure \ref{fig:dune_baseline}. The DUNE 
experiment identifies three primary scientific goals 
\cite{Abi:2020evt}:
\begin{itemize}
	\item Perform a comprehensive programme of neutrino oscillation measurements
		including measurements of \dcp{}, neutrino mass ordering, and the
		$\theta_{23}$ octant.
	\item Search for proton decay in several decay modes.
	\item Measure $\nu_e$ from a core--collapse supernova if one occurs within our
		galaxy during the lifetime of the experiment.
\end{itemize}
In addition, the experiment hopes to fulfill a significant programme of
secondary science goals:
\begin{itemize}
	\item Other accelerator based neutrino physics, such as non--standard
		interactions, sterile neutrinos, and CPT violation.
	\item Measurements of neutrino properties using atmospheric neutrinos.
	\item Dark matter searches in both the near and far detectors.
	\item A programme of neutrino interaction physics studies in the DUNE near
		detector.
\end{itemize}

To achieve these goals DUNE has opted to base the near and far detector designs
on the liquid argon time projection chamber (LArTPC) technology. The DUNE
far detector will consist of four LArTPC detectors each with 10 kt of active
liquid argon mass. This technology will have never before been used on this
scale, and therefore, there has been a significant programme of LArTPC research
and development ongoing to validate and characterise the performance of the 
technology for DUNE. 

\subsection{Liquid Argon Time Projection Chambers}
A LArTPC consists of a large volume of highly--purified liquid argon immersed in
an electric field. Charged particles traversing the liquid argon produce two
primary energy depositions, a trail of ionisation electrons along their path,
and prompt ultra--violet scintillation photons. After deposition the ionisation
electrons drift in the electric field toward the charge readout plane where they
induce electrical signals. Liquid argon is transparent to its own scintillation
light and therefore the scintillation photons can travel through the argon to be
collected in a photon detection system.The LArTPC detection principal is 
illustrated in Figure \ref{fig:lartpc}. 

\begin{figure}

	\centering

	\includegraphics[width=\textwidth]{figures/LArTPC_Concept.pdf}

	\caption
	[LArTPC detection principal.]
	{LArTPC detection principal. Figure from \cite{Abi:2020loh}.}

	\label{fig:lartpc}

\end{figure}

The details of the charge readout and photon detection systems are specific to 
each detector, but broadly speaking LArTPC detectors can be split into two 
main categories: single--phase and dual--phase. In a single--phase detector the
drifting ionisation electrons remain in the liquid argon and the signals are 
typically read out on three anode wire planes. A dual--phase LArTPC contains an
additional region of gaseous argon in which a high electric field, known as the
extraction field, is applied to extract the ionisation from the liquid 
before it is amplified and collected on a pair of anode wire planes 
\cite{Abi:2020wmh}.

\bigskip

\protodune{} is one of two large scale prototypes for the DUNE far detector
modules, which focusses on the single--phase LArTPC technology. The DUNE far
detector modules feature a modular design in which each module is built up of a
number of identical components, \protodune{} was designed to prototype the
design of many of these components at a 1:1 scale, including the anode planes,
cathode plane, and photon detectors. The \protodune{} experiment has four 
primary goals, as outlined in the Technical Design Report \cite{Abi2017}:
\begin{itemize}
	\item Prototype the production and installation procedures for the
		single--phase far detector design.
	\item Validate the design from the perspective of basic detector performance;
		this can be achieved with cosmic-ray data. 
	\item Accumulate large samples of test-beam data to understand/calibrate the
		response of the detector to different particle species.
	\item Demonstrate the long-term operational stability of the detector as part
		of the risk mitigation program ahead of the construction of the first 10 kt
		far detector module.
\end{itemize}
As such, \protodune{} represents a significant milestone in the development of
the far detector for the DUNE experiment. Its successful operation, both in a 
test--beam and with cosmic rays, provides valuable data with which to understand
reconstruction and analysis of the data that will be collected by the DUNE far 
detector.

\section{The \protodune{} Detector} \label{sec:pdsp_detector}

The \protodune{} detector is located at the Neutrino Platform at CERN along the
H4 beam line. It is a single--phase LArTPC detector with a total liquid argon 
mass of 0.77 kt, making it the largest monolithic single--phase liquid argon TPC
to be built to date. The TPC comprises the following major components, which 
are illustrated in Figure \ref{fig:pdsp_tpc}:
\begin{itemize}
	\item A cathode plane constructed of modular Cathode Plane Assemblies (CPA).
	\item Two anode planes constructed of modular Anode Plane Assemblies (APA).
	\item A photon detection system (PDS) which is integrated into the APAs.
	\item A field cage (FC), beam plug, and high voltage systems (HV).
	\item Readout electronics submerged in the liquid argon, cold electronics 
		(CE). 
\end{itemize}
The detector components are designed to be an almost exact replica of the final 
single--phase far detector modules, but the detector has an overall scaling 
factor of approximately $1:20$ in terms of total liquid argon mass 
\cite{Abi2017}.

\begin{figure}

	\centering

	\includegraphics[width=0.9\textwidth]{figures/pdsp_tpc.jpg}

	\caption
	[The main components of the \protodune{} TPC.]
	{The main components of the \protodune{} TPC. Figure from \cite{Abi2017}.}

	\label{fig:pdsp_tpc}

\end{figure}

\subsection{The Liquid Argon TPC}

The \protodune{} TPC has an active volume of 6 m (height) $\times$ 7.2 m (width,
drift direction) $\times$ 7 m (length, approximate beam direction). The cathode 
plane at the center of the active width is flanked by two anode planes which 
define two 3.6 m drift volumes. The field cage around these two drift volumes
helps to ensure a uniform electric field within the drift region.

Each anode plane is modularly constructed from three APAs which have dimensions
6 m (height) $\times$ 2.3 m (width). The APA frame holds three sets of parallel
wires on the inward and outward facing sides, these are oriented at different 
angles to enable 3D reconstruction. The first two sets of wires are induction
wires, these are electrically connected and biased such that they are
electrically transparent to the drifting ionisation; ionisation passing the
induction wires causes an induced bi--polar signal. The third set of wires are 
known as collection wires, they are not electrically connected; when drifting 
ionisation approaches the collection wires it is absorbed producing a 
uni--polar signal. In \protodune{} each set of induction wires contains 800
wires at a 4.67 mm pitch, and each set of collection wires contains 480 wires at
a 4.79 mm pitch. The wire planes from each APA are read out by CE mounted on 
the APA frame. A total of 2560 electronics channels are used to read out the 
data from each APA.  The CE amplify and digitise the signals from the wires 
before transmitting them to the Data Acquisition System (DAQ).

The cathode plane in \protodune{} consists of an array of 18 CPA modules, 2 m 
(height) $\times$ 1.2 m (width). The cathode plane is held at -180 kV to 
provide a 500 $\mbox{Vcm}^{-1}}$ drift field in each of the drift volumes. The 
field cage surrounding the drift regions ensures that the electric field is 
uniform across the detector volume. \mccorrect{How does it do this? Equally
spaced resistors I think.}

One area in which the design of \protodune{} differs from the far detector is
the inclusion of the beam plug. This is necessary to minimize interactions
between the charged particle test--beam and the cryostat before the beam enters 
the active region of the detector. A cylindrical beam plug, containing 
nitrogen gas, penetrates from the cryostat wall into the field cage at 
location of the incoming test--beam \cite{TODO}. 

\subsection{The Photon Detection System}

\subsubsection*{The Role of Light in LArTPCs}
% TODO: talk about scintialltion light production + fast and slow components

The ionisation signals in a LArTPC are slow, it takes charge milliseconds to
travel from the cathode plane to the anode plane in \protodune{}. In contrast,
scintillation photons only take on the order of nanoseconds to reach the closest
anode plane. This scintillation light plays an important role in the accurate 3D
reconstruction of interactions in the LArTPC, it provides a $t_0$. 

In a LArTPC interactions play out much quicker than the detector is able to 
record them; as such each event actually integrates over a large number of
interactions within the readout window, analogous to taking a photograph with a 
long exposure. The true time, $t_0$, of the interactions in the event cannot be 
reconstructed from the ionisation signals alone; by utilising the much faster 
scintillation signals the time of interactions can be calculated with much 
higher precision, this data can then be used to correct the position offset
caused in the ionisation signals.

\bigskip

The PDS in \protodune{} is integrated into the APAs. Ten photon detector modules
 are embedded in each APA frame between the layers of wires on each APA 
face, as shown in Figure \ref{fig:pdsp_tpc}. Three types of photon detector
module were tested in \protodune{}, two very similar module designs based on
wavelength coupling silicon photomultipliers to wavelength shifting bars, and a
third novel design know as the ARAPUCA light trap.

The majority of the photon detector modules in \protodune{}  consist of 
wavelength shifting bars coupled to silicon photomultipliers (SiPM). 
Tetraphenyl-butadiene (TPB) is used to shift the wavelength of the light 
ultra--violet to blue before the light is transmitted down the waveguide to 
the SiPMs, as illustrated in Figure \ref{fig:pdsp_pd}. The main difference 
between the two nominal designs is in the wavelength of transmission within the 
waveguide; in one case the wavelength is transmitted at the blue wavelength
produced by the TPB, in the other case the blue light from the TPB is first
absorbed in the waveguide which then produces green light which is transmitted
down the waveguide.

A small number \mccorrect{(I think only one.)} of the photon detector modules in
\protodune{} feature a novel design known as an ARAPUCA light trap. In this
design the photons are trapped in a small box through a sequence of wavelength
shifting and optical filtering, significantly increasing the photon detection
efficiency \cite{Segreto:2018jdx}. 

\mccorrect{Performance of the PD in \protodune{} is discussed in ... TODO.}


\begin{figure}

	\centering
	% TODO: make my own figure.

	\begin{subfigure}[b]{0.49\textwidth}
		\includegraphics[width=\textwidth]{figures/TODO.jpg}
		\caption{Default modules}
		\label{fig:pd_module}
	\end{subfigure}
	\hfill
	\begin{subfigure}[b]{0.49\textwidth}
		\includegraphics[width=\textwidth]{figures/TODO.jpg}
		\caption{ARAPUCA}
		\label{fig:arapuca_module}
	\end{subfigure}

	\caption
	[The operating principle of the \protodune{} photon detector modules.]
	{The operating principle of the \protodune{} photon detector modules.}

	\label{fig:pdsp_tpc}

\end{figure}


\subsection{Cold Electronics and Data Acquisition}

\section{The H4 Beam Line} \label{sec:h4}


\section{Cosmic Rays in ProtoDUNE--SP} \label{sec:pdsp_cosmic}
\subsection{The Cosmic Ray Taggers}



\section{ProtoDUNE--SP Simulation and Reconstruction} \label{sec:pdsp_sim_reco}

\subsection{Simulation}

\subsection{Reconstruction}

\section{ProtoDUNE--SP Online Monitoring System} \label{sec:pdsp_om}

