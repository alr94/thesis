\chapter{\label{ch:energyloss}Electromagnetic Energy Loss in Liquid Argon} 

%%%%%%%%%%%%%%%%%%%%%%%%%%%%%%%%%%%%%%%%%%%%%%%%%%%%%%%%%%%%%%%%%%%%%%%%%%%%%%%%
% From COS
% This chapter will cover in more detail both the theory and measurement of
% electromagnetic energy loss in liquid argon. Energy loss for both electrons and
% photons will be discussed and the implications of this for electron
% reconstruction at different energy scales will be highlighted. 
% 
% The work in this section is complete and part of this work was reported in the
% document submitted for transfer of status.
%%%%%%%%%%%%%%%%%%%%%%%%%%%%%%%%%%%%%%%%%%%%%%%%%%%%%%%%%%%%%%%%%%%%%%%%%%%%%%%%

\minitoc

The energy loss of particles in liquid argon has important implications for the
reconstruction of different particles in a LArTPC, and will be relevant for the
reconstruction algorithms developed in Chapters \ref{ch:chargeid} and 
\ref{ch:michel} of  this thesis. This chapter will cover in more detail the 
theory of electromagnetic energy loss in liquid argon, highlighting the 
important features of the energy loss for muons, electrons, and photons.

In this chapter Section \mccorect{TODO.}

\section{Electromagnetic Energy Loss in Matter}
In matter, charged particles lose energy through a number of small successive
collisions with the electrons in the material, and by radiative processes which
produce additional particles in the material. The relative importance of the
collision and radiative stopping power depends on the mass and the energy of the
particle. For most particles, which are heavy compared to the electron, 
radiative energy losses are not important until very high energies, e.g. for a 
muon they are not important until momenta of around 100 GeV. However, 
radiative energy loss become important for electrons at tens of MeV 
\cite{PhysRevD.98.030001}. As a result, different theories are used to 
describe the energy loss of heavy particles and electrons in matter.

\subsection{Energy Loss for Heavy Charged Particles}
For heavy particles such as muons at moderate energies, the mean rate of 
energy loss per unit distance is described by the Bethe equation,
\begin{equation}
	- \left< \frac{dE}{dx}\right> = K z^2 \frac{Z}{A} \frac{1}{\beta^2} 
	\left[ \frac{1}{2} \ln \frac{2 m_e c^2 \beta^2 \gamma^2 W_{max}}{I^2} -
	\beta^2 - \frac{\delta(\beta \gamma)}{2}\right].
	\label{eq:mu_stop}
\end{equation}
The constants in this equation are detailed in reference
\cite{PhysRevD.98.030001}. $Z$ and $A$ are the atomic number and mass number of
the medium, $z$ is the charge of the scattering particle, $W_{max}$ is the 
maximum energy transfer per collision, $I$ is the average excitation energy, 
and $\delta$ is a density effect correction which is relevant in solids and 
liquids. 

Three important features of the energy loss in the Bethe formula are the minimum
ionising region, the relativistic rise, and the Bragg peak, these regions are
labelled in Figure \ref{fig:muon_dedx} which shows the $dE/dx$ for muons in 
argon as a function of momentum.
\begin{figure}

	\centering

	% TODO: make me nicer, and label me
	\includegraphics[width=\textwidth]{figures/muon_dedx_argon.pdf}

	\caption
	[Stopping power as a function of energy for muons in liquid argon.]
	{ Stopping power as a function of energy for muons in liquid argon. Data from
	\cite{pdg_atomictables}.}

	\label{fig:muon_dedx}

\end{figure}

\subsubsection*{Delta Rays}
Another feature of the electromagnetic energy loss of heavy particles that
impacts reconstruction in LArTPCs is delta rays. Delta rays are energetic
electrons which are knocked out of their atoms when they collide with the heavy
particle, in liquid argon detectors these electrons are seen as small electron 
tracks which protrude from muon tracks. 

\subsection{Energy Loss for Electrons}
Electrons and positrons undergo different electromagnetic scattering processes
in matter, M{\o}ller scattering and Bhabha scattering respectively 
\cite{TODO}. These processes, which dominate electron and positron energy loss 
at low energies, have different cross sections which modify the energy loss in 
each case. At higher energies, radiative processes such as bremsstrahlung 
dominate. The two components of the electron stopping power are known as the 
collision stopping power and the radiative stopping power.

\subsubsection*{Collision Stopping Power}
The collision stopping power of electrons and positrons is calculated with a
similar method to the heavy particle stopping power, where individual collisions
are considered in succession. The main difference in the calculations is the
cross sections used in the calculations, for electrons the M{\o}ller scattering
cross section is used, and for positrons Bhabha scattering is considered. Due to
the electrons and positrons having the same mass as their targets, the maximum 
energy transfer in a single collision, $W_{max}$, is the total kinetic energy. 
However, this value is halved for the case of electrons, due to the convention 
of calculating the stopping power for the final state electron with higher 
kinetic energy.

The stopping power based on M{\o}ller scattering of electrons gives,
\begin{align*}
	- \left< \frac{dE}{dx} \right> = \frac{1}{2} K \frac{Z}{A} \frac{1}{\beta^2}
	\bigg[ &\ln \frac{m_e c^2 \beta^2 \gamma^2 \left\{ m_e c^2 (\gamma - 1) / 2
	\right\} }{I^2} + (1 - \beta^2) \\
	&- \frac{2\gamma - 1}{\gamma} + \frac{1}{8} 
	\left(\frac{\gamma - 1}{\gamma}\right)^2 - \delta \bigg],
\end{align*}
while Bhabha scattering, which governs the positron stopping power, gives,
\begin{align*}
	- \left< \frac{dE}{dx} \right> = \frac{1}{2} K \frac{Z}{A} \frac{1}{\beta^2}
	\bigg[ &\ln \frac{m_e c^2 \beta^2 \gamma^2 \left\{ m_e c^2 (\gamma - 1) 
	\right\} }{2 I^2} 
	+ 2 \ln 2  \\ &-\frac{\beta^2}{12} \left(23 + \frac{14}{\gamma + 1} +
	\frac{10}{(\gamma + 1)^2} + \frac{4}{(\gamma + 1)^3}\right) - \delta \bigg],
\end{align*}
where the terms have the same meanings as in Equation \ref{eq:mu_stop}
\cite{PhysRevD.98.030001}.

\subsubsection*{Radiative Stopping Power}
Above a few tens of MeV, electrons lose most of their energy through the 
emission of bremsstrahlung photons. Detailed discussion of the energy loss due
to bremsstrahlung emission is beyond the scope of this thesis, detailed
discussions are provided by \cite{PhysRevD.98.030001, Tsai:1973py}. Here, we 
discuss a simplified model which highlights the important factors relevant for 
the work in this thesis.

At high energies, where the radiative energy loss is dominant, the energy of 
the electron can be approximated as an exponential decay over a length scale 
known as the radiation length, $X_0$,
\begin{equation*}
	E = E_0 \; e^{-x/X_0}.
\end{equation*}
In this approximation, the energy loss per unit distance due to bremsstrahlung 
is, 
\begin{equation*}
	- \left( \frac{dE}{dx} \right)_{brem} = \frac{E}{X_0}.
\end{equation*}
The radiation length, $X_0$, can be parametrised as,
\begin{equation}
	\begin{gathered}
		\frac{1}{X_0} = 4 \alpha r_e^2 \frac{N_A}{A} \left\{ Z^2 \left[L_{rad} - f(Z)\right] + Z
		L^\prime_{rad} \right\} \\
		\begin{split}
			f(Z) = \alpha^2 Z^2 \bigg[ &\frac{1}{1 + \alpha^2 Z^2} + 0.20206 - 0.0369
			\alpha^2 Z^2 \\ &+ 0.0083 \alpha^4 Z^4 -0.0002 \alpha^6 Z^6 \bigg],
		\end{split}
	\end{gathered}
	\label{eq:rad_length}
\end{equation}
where $L_{rad}$ and $L_{rad}^\prime$ are the so--called radiation logarithms,
which depend on the atomic number of the material\cite{Tsai:1973py}.

\subsubsection*{Critical Energy}
The critical energy is often defined as the energy at which the collision and 
radiative stopping power are equivalent, other definitions are also used, such
as the definition by Rossi, the energy at which the ionisation loss per
radiation length is equal to the electron energy \cite{TODO}. Rossi's 
definition is equivalent to using the approximate $dE/dx$ calculated 
above\cite{PhysRevD.98.030001}. The value of the critical energy has important 
implications for reconstruction algorithms, because different approaches are 
often required above and below the critical energy. 

The critical energy is slightly different for electrons and positrons, in 
liquid argon they are both around 32 MeV, based on the Rossi definition 
\cite{pdg_atomictables}. This can be seen in Figure \ref{fig:electron_dedx}, 
which shows the total electron stopping power in liquid argon, in addition to 
the collision and radiative components which make up the total stopping 
power. 

\begin{figure}

	\centering

	\includegraphics[width=\textwidth]{figures/electron_dedx_argon.pdf}

	\caption
	[Stopping power as a function of energy for electrons in liquid argon.]
	{Stopping power as a function of energy for electrons in liquid argon. Data
		from \cite{estar}.}

	\label{fig:electron_dedx}

\end{figure}

\subsection{Energy Loss for Photons}
A number of processes contribute to the energy loss of photons in matter, brief 
descriptions of the main processes are given below.

\subsubsection*{Photoelectric Effect}
The photoelectric effect occurs when a photon collides with an atom, X, the
photon is absorbed and electron is emitted from the atom. As a result, the atom 
is ionised. 
\begin{equation*}
	\gamma + X \rightarrow e^- + X^+
\end{equation*}

\subsubsection*{Compton Scattering}
Compton scattering occurs when a photon scatters incoherently from an electron
within an atom. The electron is typically liberated from the atom, and the
photon loses some of it's energy.
\begin{equation*}
	\gamma + e^- \rightarrow \gamma + e^-
\end{equation*}

\subsubsection*{Pair Production}
Pair production is the production of an electron positron pair, in the 
vicinity of an external electric field. During this process, the photon is
destroyed to produce the electron positron pair. In matter, the electric field
could be provided by either the electrons in the atom, or the nucleus of the
atom.
\begin{equation*}
	\gamma \rightarrow e^+ + e^-
\end{equation*}

\bigskip

The cross section for these effects vary as a function of photon energy. The 
cross sections for each process in liquid argon, as well as the total photon 
cross section, are given in Figure \ref{fig:photon_xsec}. The Compton scattering
cross section is dominant from around 0.1 MeV to 10 MeV, after which the pair
production cross section dominates. 

\begin{figure}

	\centering

	\includegraphics[width=\textwidth]{figures/photon_xsec.pdf}

	\caption
	[Photon interaction cross sections in liquid argon.]
	{ Photon interaction cross sections in liquid argon. Data from
	\cite{photon_xsec}.}

	\label{fig:photon_xsec}

\end{figure}

\subsubsection*{Photon Mean Free Path}
The mean free path of a photon is defined as the distance travelled by the
photon before it interacts with the material. The mean free path for photons
has two main components for photons in the MeV range, which are due to Compton
scattering and pair production. The mean free path is given by $\lambda = 1 / (n
\sigma)$, where $n$ is the number density of targets and $\sigma$ is the cross
section per target. The contribution to the mean free path from pair production
is related to the radiation length for electrons, $X_0$ from Equation
\ref{eq:rad_length}, by $\lambda_{PP} = (9/7) \; X_0$ \cite{PhysRevD.98.030001}.
The mean free path for photons in the MeV range is shown in Figure
\ref{fig:photon_mfp}, along with the main components. 

\begin{figure}

	\centering

	\includegraphics[width=\textwidth]{figures/photon_mfp.pdf}

	\caption
	[Photon mean free path in liquid argon.]
	{Photon mean free path in liquid argon. Data from \cite{photon_xsec}.}

	\label{fig:photon_mfp}

\end{figure}
