\chapter{\label{ch:conclusions}Conclusions} 

%%%%%%%%%%%%%%%%%%%%%%%%%%%%%%%%%%%%%%%%%%%%%%%%%%%%%%%%%%%%%%%%%%%%%%%%%%%%%%%%
% From COS
% This chapter will summarise the work presented in the thesis and provide
% concluding remarks on the implications of the results for future analyses in
% LArTPC experiments
%%%%%%%%%%%%%%%%%%%%%%%%%%%%%%%%%%%%%%%%%%%%%%%%%%%%%%%%%%%%%%%%%%%%%%%%%%%%%%%%.

\minitoc

The Deep Underground Neutrino Experiment (DUNE) is a next generation neutrino
oscillation experiment, which aims to make precision measurements of neutrino
oscillation probabilities, in order to search of CP--violation in the lepton
sector. In addition, DUNE provides an opportunity to study the mechanisms
involved in supernova bursts through the measurement of supernova neutrinos. 
To achieve these goals, the charged particles in the DUNE detectors will have to
be accurately reconstructed, in order to allow the details of the neutrino
interactions to be determined. This thesis presented the results of the
development of two reconstruction algorithms in the \protodune{} detector, which
is a surface level proto--type for the DUNE far detector modules. 

The development of a hit tagging algorithm for track, shower, and Michel
electron hit classification was discussed in Chapter \ref{ch:chargeid}. This 
algorithm is based on convolution neural networks, with a small patch 
approach. In terms of track--shower discrimination, the algorithm was shown 
to reduce the false--positive rate for all particle species, with a reduction 
of more than a factor of 10 in some cases. In addition it was shown to have a 
small discrepancy, of around 1-3\%, for hit--by--hit selection. This algorithm 
is now seeing significant use within ongoing \protodune{} data analyses. 

% TODO: Discuss MicroBooNE \cite{Acciarri2017}, ICARUS \cite{Amoruso:2003sw}, 
% and LArIAT \cite{Foreman_2016} results somewhere

A sample of Michel electrons was used to analyse the low energy electron 
reconstruction capabilities of the \protodune{} detector. A sample of Michel
electrons was selected with over 98\% purity, and around 6\% efficiency, by
selecting clusters of hits with a high Michel electron score from the neural
network discussed in Chapter \ref{ch:chargeid}.  

The selected Michel electron events, were reconstructed with a convolutional
neural network designed to perform semantic segmentation on the input images. 
This network demonstrated a high hit tagging purity and completeness of
\mccorrect{TODO\%} and \mccorrectTODO\%} respectively, which is a marginal improvement over the Michel 
electron reconstruction algorithm developed for the MicroBooNE experiment. 

The reconstructed Michel electron properties were compared between data and 
simulation, and the Michel electron energy resolution and bias were estimated 
to be \mccorrect{TODO\%} and \mccorrect{TODO\%}. The measured energy resolution
and bias are \mccorrect{TODO} to 
Michel electron analyses in MicroBooNE, \mccorrect{TODO}. 

% TODO: talk about proposed energy resolution from TDR \cite{Abi:2020evt}
The measured energy resolution for Michel electrons, is significantly lower than
the proposed supernova neutrino energy resolution for the DUNE far detector. In
addition, the maximum possible theoretical energy resolution based on ionisation
energy deposition was investigated, which was found to be \mccorrect{TODO\%} based on 
\protodune{} simulation. Therefore, it is clear that, in order to achieve the 
proposed energy resolution for low energy electrons, the ionisation energy 
alone in insufficient, and additional energy needs to be recovered. Analyses of
ionisation energy have been performed by the MicroBooNE, 
A combined
analysis of ionisation energy and oscillation

\mccorrect{TODO: Future improvements.}
