\chapter{\label{ch:conclusion}Conclusion} 

%%%%%%%%%%%%%%%%%%%%%%%%%%%%%%%%%%%%%%%%%%%%%%%%%%%%%%%%%%%%%%%%%%%%%%%%%%%%%%%%
% From COS
% This chapter will summarise the work presented in the thesis and provide
% concluding remarks on the implications of the results for future analyses in
% LArTPC experiments
%%%%%%%%%%%%%%%%%%%%%%%%%%%%%%%%%%%%%%%%%%%%%%%%%%%%%%%%%%%%%%%%%%%%%%%%%%%%%%%%.

\minitoc

The Deep Underground Neutrino Experiment (DUNE) is a next generation neutrino
oscillation experiment, which aims to make precision measurements of neutrino
oscillation probabilities, in order to search for CP--violation in the lepton
sector. In addition, DUNE provides an opportunity to study the mechanisms
involved in supernova bursts through the measurement of supernova neutrinos. 
To achieve these goals, the charged particles in the DUNE detectors will have to
be accurately reconstructed, in order to allow the details of the neutrino
interactions to be determined. This thesis presented the results of the
development of two reconstruction algorithms in the \protodune{} detector, which
is a surface level proto--type for the single phase DUNE far detector modules. 

In this thesis, the development of a hit tagging algorithm for track, shower, 
and Michel electron hit classification was discussed in Chapter 
\ref{ch:chargeid}. This algorithm is based on convolution neural networks, 
with a small patch approach. In terms of track--shower discrimination, the 
algorithm was shown to reduce the false--positive rate for all particle 
species, with a reduction of more than a factor of 10 in some cases, when
compared to the previous algorithm. In addition, this algorithm was shown to 
give a good agreement, of around 1--3\%, between data and simulation for 
hit--by--hit event selection. This algorithm is now seeing significant use 
within ongoing \protodune{} data analyses, including pion cross--section 
analyses and detector calibrations.  Additionally, this network was used as 
part of the Michel electron event selection algorithm, which was discussed in 
Chapter \ref{ch:michel} of this thesis.

A sample of Michel electrons from the \protodune{} detector was used to study
Michel electron reconstruction in LArTPCs, and to estimate the energy resolution
and bias of \protodune{} for electrons in the tens of MeV range. Michel electron
events where selected based on a new algorithm, which combined the clustering of
Pandora with the output of the hit classification CNN to select clusters with a
high fraction of Michel--like hits. This algorithm was shown to agree well
between data and simulation, with around a 2\% difference between the selected
fraction in data and simulation. The simulation was used to estimate the purity
and efficiency of the event selection algorithm, which were found to be around
98\% and 6\% respectively. This represents an improvement in both purity and
efficiency over similar LArTPC Michel electron studies from other experiments,
such as MicroBooNE.

The Michel electron sample was used to investigate a novel reconstruction
technique, which was based on the semantic segmentation of images with a
convolutional neural network. A U--ResNet architecture was used to perform the
segmentation, and the results were promising, however, more work is required to
improve the performance of this algorithm across the whole energy range. 

In the 0--25 MeV region, where the performance of the U--ResNet algorithm was 
good, the ionisation energy resolution for Michel electrons based on this 
algorithm was found to be around 10--12\%, and the bias was less than 
5\%. This represents an improvement over the measured Michel electron ionisation
energy resolution of MicroBooNE, which is the closest comparison to \protodune{}
in terms of Michel electron reconstruction. Above 25 MeV, the reconstruction 
algorithm began to develop a significant tail in the fractional difference 
distribution. This tail represents the tendency of the algorithm to 
under--reconstruct the Michel electron ionisation energy, and is a result of a 
drop in hit selection efficiency above 25 MeV. Some additional work
is required to improve the performance of this algorithm, but the excellent
performance in the low energy region shows that this is a promising approach. A
similar approach could be developed, in the future, for other low--energy 
reconstruction tasks in LArTPCs, such as the reconstruction of supernova 
neutrinos in the DUNE far detector.

The DUNE experiment will use a liquid argon time projection chamber to make 
precision studies of neutrino oscillations, supernova neutrinos, and beyond 
the standard model phenomena. The development of effective reconstruction 
algorithms and understanding the detector response are both crucial components 
of understanding particle interactions in the DUNE detector. This thesis has 
presented the results of various studies into electromagnetic interactions in 
the \protodune{} detector, focussing on the application of convolutional 
neural networks to event reconstruction in LArTPCs. The results of these 
analyses are currently being used to assist in physics measurements based on 
\protodune{} data, and can also be used to guide further development of 
reconstruction strategies for LArTPC detectors.
