\chapter{\label{ch:2-neutrinophysics}Neutrino Physics} 

%%%%%%%%%%%%%%%%%%%%%%%%%%%%%%%%%%%%%%%%%%%%%%%%%%%%%%%%%%%%%%%%%%%%%%%%%%%%%%%%
% From my COS report

% This chapter will cover details of relevant neutrino physics for the thesis 
% project. The theoretical aspects of neutrino physics will be reviewed 
% including the theory of neutrino oscillations in the PMNS framework. 
% Neutrino production will also be discussed including details of neutrino 
% production in supernovae and the role neutrinos can have in understanding 
% the mechanics of a supernova burst
 
% The work for this section is ongoing as it is associated with the analysis 
% work in the other sections, it is expected to be completed by the end of 
% December 2019.
%%%%%%%%%%%%%%%%%%%%%%%%%%%%%%%%%%%%%%%%%%%%%%%%%%%%%%%%%%%%%%%%%%%%%%%%%%%%%%%%

\minitoc

Despite being the second most abundant particle in the universe, neutrinos are 
some of the most elusive. Postulated by Wolfgang Pauli in 1930 \ref{pauli}, to 
explain the continuous energy distribution of electrons emitted in 
\(\beta\)-decays, it would be over 25 years before Cowen and Reines would detect 
neutrinos in the vicinity of a nuclear reactor \ref{reines_cowen}. Since then
our understanding of neutrino physics has expanded, three unique neutrino 
flavours have been discovered, and the discovery of neutrino oscillations makes 
neutrinos the only fundamental particle capable of time dependent flavour 
changes. The theory explaining neutrino oscillations, the 
Pontecorvo–Maki–Nakagawa–Sakata (PMNS) theory, has become well established by 
experimental measurements and leads to the possibility of CP--violation in the 
leptonic sector, a prerequisite for many explanations of the observed matter 
antimatter--asymmetry of the modern universe. 

This chapter will review our current understanding of neutrino physics and the 
role of neutrinos in the Standard Model (SM).  Theoretical aspects of neutrino 
physics will be reviewed in section \ref{nu_sm}, with a more detailed 
discussion of the theory of neutrino oscillations in section \ref{nu_osc}.  
Section \ref{nu_prod} will give details of neutrino interactions and production. 
The production of neutrinos in supernova bursts will be discussed in section 
\ref{nu_sn}, and the role of neutrinos in understanding supernovae will be 
highlighted.

% TODO: decide on historical vs theoretical overview or both
\section{A Brief History of Neutrino Physics} \label{nu_sm}
% TODO: References

When the energy spectra of electrons in \(\beta\)--decays where first measured
the results posed a problem; at the time the \(\beta\)--decay interaction was
believed to lead to a two--body final state and, as such, the emitted electrons
should be released with a single well defined energy. Therefore the observation
of a broad spectrum of electron energies appeared to disobey conservation of
energy, in 1930 Pauli proposed a solution to this problem by postulating an 
additional particle that was evading detection but sharing the released energy 
with the electron, thus preserving conservation of energy. Despite Pauli's 
belief that he had proposed the existence of something that "cannot be 
detected", in 1956 Cowen and Reines would announce that they had observed 
positrons produced by the inverse beta decays of neutrinos from a nuclear 
reactor. 

After the first observation of neutrinos the progression in experimental 
neutrino physics became more rapid. TODO

\section{Neutrinos in the Standard Model} \label{nu_sm}

In the Standard Model (SM) Neutrinos, and their interactions, are governed by 
the electroweak (EW) theory, a gauge theory on the \(\mbox{SU}(2) \times 
\mbox{U}(1)\) group. In this theory, neutrinos transform along with charged 
leptons as part of the left handed fermion doublets
\begin{equation}
	\Psi = 
	\begin{cases} 
		\left( \begin{array}{c} \nu_i \\ l^-_i \end{array} \right), \left( 
			\begin{array}{c} u_i \\ d'_i \end{array} \right)
	\end{cases}
\end{equation}
where \(d'_i = \sum_j V_{ij} d_j\), and \(V\) is the Cabibbo-Kobayashi-Maskawa 
(CKM) mixing matrix.

\section{Neutrino Oscillations} \label{nu_osc}


\section{Neutrino Interactions} \label{nu_prod}


\section{Supernova Neutrinos} \label{nu_sn}

