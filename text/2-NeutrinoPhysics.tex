\chapter{\label{ch:2-neutrinophysics}Neutrino Physics} 

%%%%%%%%%%%%%%%%%%%%%%%%%%%%%%%%%%%%%%%%%%%%%%%%%%%%%%%%%%%%%%%%%%%%%%%%%%%%%%%%
% From my COS report

% This chapter will cover details of relevant neutrino physics for the thesis 
% project. The theoretical aspects of neutrino physics will be reviewed 
% including the theory of neutrino oscillations in the PMNS framework. 
% Neutrino production will also be discussed including details of neutrino 
% production in supernovae and the role neutrinos can have in understanding 
% the mechanics of a supernova burst
 
% The work for this section is ongoing as it is associated with the analysis 
% work in the other sections, it is expected to be completed by the end of 
% December 2019.
%%%%%%%%%%%%%%%%%%%%%%%%%%%%%%%%%%%%%%%%%%%%%%%%%%%%%%%%%%%%%%%%%%%%%%%%%%%%%%%%

% TODO: References

% \minitoc

Despite being one of the most abundant particle in the universe, neutrinos are 
some of the most elusive; due to the fact that neutrinos can only interact via
the weak interaction. The history of neutrino physics is therefore strongly
connected to the discovery and study of weak interactions. Measurements by
Chadwick in 1914 showed that the energy spectrum of electrons released in
\(\beta\)--decays was continuous, this is in contrast to discrete spectra
observed in \(\alpha\) and \(\gamma\) decays, and seemingly violates
conservation of energy under the assumption of a two--body final state which was
expected at the time. In order to solve this problem, Pauli postulated that the
continuous energy spectrum could be explained if the energy released in a 
\(\beta\)--decay could be shared with an additional neutral weakly interacting 
fermion which Pauli named the neutron. Fermi later renamed Pauli's fermion to
the neutrino, after Chadwick discovered the neutron in 1932. Despite claims that
neutrinos might never be detected, neutrinos have now been discovered and they
have been found to have a number of interesting properties which were not
anticipated when neutrinos were first postulated. This chapter will detail some 
of the history and theory of neutrino's and their interactions.

In this chapter, Section \ref{nu_hist} will give a brief historical overview of 
neutrino physics, Section \ref{nu_osc} will introduce neutrino oscillations
and the theory used to describe them, while Section \ref{nu_prod} will discuss
neutrino interactions in matter as well as neutrino production. Finally Section
\ref{nu_sn} will discuss the production of neutrinos in supernovae, as well as
the role they could play in understanding supernovae.

% TODO: decide on historical vs theoretical overview or both
\section{A Brief History of Neutrino Physics} \label{nu_hist}

The first attempt to incorporate the neutrino into a theoretical model came in
1934 when Fermi presented his theory of \(\beta\)--decay, in this theory the 
neutrino takes part in a four--point interaction with the other components of 
the \(\beta\)--decay interaction. The incredible success of this theory in
explaining the observed properties of \(\beta\)--decays provided strong evidence 
for the neutrinos existence, however, in 1934 after using Fermi's  theory to 
predict the strength of neutrino interactions, H. Bethe and R. Peierls found 
that the interactions were so weak that they might never be observed, a 
prediction that held true for over 20 years.% TODO: refs

The first breakthrough in experimental neutrino physics would come in 1956. F.
Reines and C. Cowan were attempting to measure positrons produced in inverse 
\(\beta\)--decay interactions,
\begin{equation}
	\bar{\nu_e} + p \rightarrow n + e^+.
\end{equation}
A detector containing 1400 litres of liquid scintillator was used to measure the 
large flux of electron anti--neutrinos in the vicinity of the Savannah River 
nuclear reactor. They observed a large increase in the rate of positron events 
when the reactor was on when compared to when the reactor was switched off, the 
first experimental evidence for the existence of neutrinos. % TODO: ref

The discovery of the electron neutrino opened the door to answer questions of 
neutrino flavour. As neutrinos are produced alongside a charged lepton it is 
natural to compare the properties of neutrinos with their partners in the weak
interaction. At the time of the discovery of the neutrino there were two known
charged leptons, the electron and the muon, and so physicists asked whether the
neutrinos produced alongside muons are different from those produced alongside
electrons. In 1962, Lederman et al discovered the muon neutrino at Brookhaven
National Laboratory; by creating a beam of muon associated neutrinos using 
decaying pions, and observing the leptons produced in neutrino interactions 
after all other particles had been absorbed. They found that only muons where 
produced in the resulting neutrino interactions, and therefore the neutrinos 
produced were only ever associated with a muon, which shows that neutrinos are 
produced with a distinct flavour in weak interactions.

With the discovery of the tau--lepton in 1977, it was expected that there 
should be an associated tau neutrino. However, it would be a long time before
the tau neutrino would be detected. This happened in 2001 when the DONUT
experiment ... TODO



%%%%%%%%%%%%%%%%%%%%%%%%%%%%%%%%%%%%%%%%%%%%%%%%%%%%%%%%%%%%%%%%%%%%%%%%%%%%%%%%
% TODO
% Neutral Current & gargamelle
% nu tau
% 3 nu's at LEP
% nu osc
% atmos osc
% solar osc
% reactor osc
% tau osc
%%%%%%%%%%%%%%%%%%%%%%%%%%%%%%%%%%%%%%%%%%%%%%%%%%%%%%%%%%%%%%%%%%%%%%%%%%%%%%%%

% \section{Neutrinos in the Standard Model} \label{nu_sm}

\section{Neutrino Oscillations} \label{nu_osc}

\section{Neutrino Interactions} \label{nu_prod}

\section{Supernova Neutrinos} \label{nu_sn}
